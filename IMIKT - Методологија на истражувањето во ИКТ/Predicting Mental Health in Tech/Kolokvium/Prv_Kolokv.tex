\documentclass[12pt]{article}
\usepackage[utf8]{inputenc}
\usepackage[macedonian]{babel}

\title{Колку точно можеме да предвидиме ментални здравствени состојби кај вработени кои работат од далечина, користејќи податоци од анкета?}
\author{
    Никола Иванов, 213212 \\ 
    Берат Ахметај, 216130 \\ 
    Илија Трајковски, 213188
}
\date{}

\begin{document}

\maketitle

\section*{1. Работен наслов на проектот и тим}
\textbf{Работен наслов:}  
Колку точно можеме да предвидиме ментални здравствени состојби кај вработени кои работат од далечина?  

\textbf{Тим:} \newline
Никола Иванов, 213212 \newline
Берат Ахметај, 216130 \newline
Илија Трајковски, 213188


\section*{2. Тема}
Предвидување на ментални здравствени состојби кај вработени кои работат од далечина, користејќи податоци од анкета за нивните нивоа на стрес, социјална изолација и задоволство со работата. \newline
\textbf{Област:} Податочна Анализа

\section*{3. Мотив за истражување}
Ова истражување има за цел да помогне на организациите да ги идентификуваат факторите што најмногу влијаат врз менталното здравје, со цел да воведат подобри практики и ресурси за поддршка на вработените.  

\section*{4. Истражувачки прашања}
\begin{itemize}
    \item Дали нивоата на стрес и социјална изолација можат да предвидат ментални здравствени состојби кај вработени кои работат од далечина?
    \item Колку е значајно задоволството од работата за предвидување на менталното здравје на вработените?
\end{itemize}

\textbf{Образложение:}  
Овие прашања се релевантни бидејќи работата од далечина гледа пораст во дигиталната ера со тоа и менталното здравје. Разбирањето на овие релации може да помогне во развој на политики за подобрување на благосостојбата на работното место.  

\section*{5. Хипотези}
\begin{enumerate}
    \item Поголемите нивоа на стрес и социјална изолација се поврзани со поголема веројатност за ментални здравствени состојби.
    \item Високото задоволство од работата е поврзано со помала веројатност за ментални здравствени состојби.
\end{enumerate}

\section*{6. Листа на трудови од областа}
\begin{itemize}
    
\end{itemize}

\textbf{Како се разликува предлог истражувањето:}  
 

\section*{7. Предлог Методологија}
\textbf{Тип:} Квантитативно истражување. \newline
\textbf{Опис:} Користење на алгоритми (логистичка регресија и машини со потпорни вектори) за анализа и предвидување. \newline 
\textbf{Образложение:} Преку користење на овие алгоритми, може поефикасно да ги најдеме значајните фактори кои влијаат во вреностите. \newline 

\section*{8. Опис на експериментот}
\textbf{Метод:}  
Логистичка регресија и SVM Анализа на податоци собрани преку анкети со 5,000 записи за вработени ширум светот и база на податоци на вработени преку далечина.  

\textbf{Релевантност:}  


\end{document}
