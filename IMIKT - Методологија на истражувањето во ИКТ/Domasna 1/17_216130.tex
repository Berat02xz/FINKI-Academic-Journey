\documentclass[12pt]{article}
\usepackage[utf8]{inputenc}
\usepackage[macedonian]{babel}
\usepackage[style=numeric,sorting=none]{biblatex} % Load biblatex package
\usepackage{hyperref} % For clickable links
\addbibresource{references.bib} % Add the bibliography file

\title{Анализа на труд: Стрес на работа и осаменост меѓу вработените на канцелариски работни места за време на пандемијата на COVID-19 во Јапонија}
\author{}
\date{}

\begin{document}

\maketitle

\section*{1. Истражувачко прашање}
Истражувачкото прашање на овој труд е: Како стресот на работа и осаменоста се поврзани кај вработените кои работат на канцелариски работни места, со фокус на влијанието на работата од далечина?

\section*{2. Хипотези}
\begin{itemize}
    \item Работата од далечина е поврзана со зголемени нивоа на осаменост кај вработените.
    \item Работата од далечина има влијание на стресот на работа и нивото на осаменост.
    \item Поддршката од колеги и менаџери ја намалува осаменоста кај вработените, вклучително и оние кои работат од далечина.
\end{itemize}

\section*{3. Методологија}
Овој труд е дел од проектот "Collaborative Online Research on the Novel-coronavirus and Work (CORoNaWork)" во Јапонија. Податоците се собрани од 13,468 работници кои работат на канцелариски работни места. Осаменоста беше оценувана преку едноставно прашање, а стресот на работа беше проценуван со користење на "Job Content Questionnaire (JCQ)". За анализа на податоците, беше применета мултипла логистичка регресија.

\section*{4. Резултати}
Резултатите покажаа дека вработените кои работат од далечина 4 или повеќе дена неделно имаат поголема веројатност да пријават чувства на осаменост во споредба со оние кои не работат од далечина (прилагоденото односно со шансата: 1.23). Работата од далечина не го објасни интеракцијата помеѓу скалите на JCQ и осаменоста. Меѓу вработените кои работат од далечина, поддршката од колегите и менаџерите беше силно поврзана со чувството на осаменост.

\section*{5. Поврзаност со нашето истражување}
Овој труд е поврзан со нашето истражување бидејќи исто така се фокусира на стресот на работа и осаменоста кај вработени кои работат од далечина. Нашето истражување ги разгледува факторите кои влијаат на менталното здравје на вработените оддалечено, како што се стресот, социјалната изолација и задоволството од работата. Овој труд обезбедува вредни податоци за врската помеѓу работата од далечина и менталното здравје, што ја поткрепува нашата хипотеза дека комуникацијата и поддршката играат клучна улога во намалувањето на осаменоста и стресот.

\section*{6. Цитирање на трудот}

Цитирање на трудот според \texttt{biblatex} формат:

\begin{quote}
Мијаке, Ф., Одгерел, Ч.О., Хино, А., Икегами, К., Нагада, Т., Татеиши, С., Цуџи, М., Мацуда, С., Ишимару, Т. Стрес на работа и осаменост меѓу вработените на канцелариски работни места за време на пандемијата на COVID-19 во Јапонија: фокус на работата од далечина. \textit{Environ Health Prev Med}. 2022;27:33. doi: 10.1265/ehpm.22-00107. PMID: 35965100; PMCID: PMC9425057.
\end{quote}

\section*{7. Линкови}
целосната PDF верзија на истражувањето: \href{https://pmc.ncbi.nlm.nih.gov/articles/PMC9425057/pdf/ehpm-27-033.pdf}{целосен PDF}.

веб-страницата на PubMed: \href{https://pubmed.ncbi.nlm.nih.gov/35965100/}{PubMed страница}.

\newpage

\printbibliography % This will print the bibliography at the end of your document

\end{document}
