\documentclass[12pt]{article}
\usepackage[utf8]{inputenc}
\usepackage[macedonian]{babel}

\title{Прв Колоквиум }

\date{19.11.2024}

\begin{document}

\maketitle

\section*{1. Работен наслов на проектот и тим}
\textbf{Работен наслов:}  
Колку точно можеме да предвидиме ментални здравствени состојби кај вработени кои работат од далечина?  

\textbf{Тим:} \newline
Никола Иванов, 213212 \newline
Берат Ахметај, 216130 \newline
Илија Трајковски, 213188


\section*{2. Тема}
Предвидување на ментални здравствени состојби кај вработени кои работат од далечина, користејќи податоци од анкета за нивните нивоа на стрес, социјална изолација и задоволство со работата. \newline
\textbf{Област:} Податочна Анализа

\section*{3. Мотив за истражување}
Ова истражување има за цел да помогне на организациите да ги идентификуваат факторите што најмногу влијаат врз менталното здравје, со цел да воведат подобри практики и ресурси за поддршка на вработените.  

\section*{4. Истражувачки прашања}
\begin{itemize}
    \item Дали нивоата на стрес и социјална изолација можат да предвидат ментални здравствени состојби кај вработени кои работат од далечина?
    \item Колку е значајно задоволството од работата за предвидување на менталното здравје на вработените?
    \item Дали редовната комуникација со колеги и менаџери може да има заштитен ефект врз менталното здравје на вработените кои работат од далечина?
\end{itemize}

\textbf{Образложение:}  
Овие прашања се релевантни бидејќи работата од далечина гледа пораст во дигиталната ера со тоа и менталното здравје. Разбирањето на овие релации може да помогне во развој на политики за подобрување на благосостојбата на работното место.  

\section*{5. Хипотези}
\begin{enumerate}
    \item Поголемите нивоа на стрес и социјална изолација се поврзани со поголема веројатност за ментални здравствени состојби.
    \item Високото задоволство од работата е поврзано со помала веројатност за ментални здравствени состојби.
    \item Поголемата фреквенција на комуникација со колеги и менаџери е позитивно поврзана со менталното здравје на вработените.
    \item Хибридното работење, кое комбинира работа од дома и од канцеларија, може да придонесе за подобро ментално здравје и поголемо задоволство од работата, бидејќи овозможува поголем баланс меѓу професионалниот и личниот живот.
    \\*
    \\*
    \\*
\end{enumerate}

\section*{6. Листа на трудови од областа}
\begin{itemize}
    \item \textbf {"The Impact of Remote Work on Employee Productivity and Well-being: A Comparative Study of Pre- and Post-COVID-19 Era"}
    Овој труд ги анализира промените во продуктивноста и балансот меѓу работата и приватниот живот кај вработени кои работат од далечина. Релевантен е бидејќи обезбедува податоци за различни фактори кои влијаат врз продуктивноста и задоволството, што ги надградува нашите истражувачки прашања.
    \item \textbf {"Social Isolation and Stress as Predictors of Productivity Perception and Remote Work Satisfaction during the COVID-19 Pandemic”}
    Овој труд потврдува дека зголемената изолација доведува до повисоки нивоа на стрес, додека честа комуникација со колеги и менаџери го намалува стресот и ја подобрува благосостојбата на вработените. Ова се совпаѓа со нашите хипотези, особено за поврзаноста на стресот и социјалната изолација со менталните здравствени состојби, како и значењето на комуникацијата за подобрување на менталното здравје.Релевантен е бидејќи нашето истражување се фокусира на специфични предиктивни модели за идентификација на ризиците.
    \item \textbf {"Machine learning approaches to predict mental health outcomes in corporate environments."}
    Нашето истражување е поврзано со овој труд, бидејќи и двата се фокусираат на предвидување на ментални здравствени состојби преку анализа на фактори како стресот и социјалната изолација. Сепак, нашето истражување е насочено кон вработени кои работат од далечина, за разлика од нивниот поширок корпоративен фокус. Додека тие користат различни машински методи, ние користиме логистичка регресија (LR) и машини со потпорни вектори (SVM) за анализа на анкетни податоци, со акцент на фактори како хибридно работење, комуникација и задоволство од работата. Ваквиот труд ни обезбедува методолошка основа, додека нашето истражување додава уникатна перспектива во контекстот на далечинско работење.
    \\*
    \\*
    \\*
\end{itemize}

\textbf{Како се разликува предлог истражувањето:} \newline
 Нашето истражување се издвојува со фокусирање на специфичен контекст – вработени кои работат од далечина, што е растечки тренд во модерното работно опкружување. За разлика од постоечките трудови кои генерално се занимаваат со менталното здравје во корпоративни средини, нашето истражување таргетира клучни фактори како што се стресот, социјалната изолација и задоволството од работата, користејќи ги како предиктори за ментални здравствени состојби. Дополнително, воведуваме анализа на хибридното работење, со цел да истражиме како оваа структура може да влијае на балансот меѓу работата и приватниот живот.Методолошки, нашето истражување применува напредни алгоритми како логистичка регресија и машини со потпорни вектори за анализа на голем сет податоци од анкети, што обезбедува прецизна идентификација на ризичните фактори.
 

\section*{7. Предлог Методологија}
\textbf{Тип:} Квантитативно истражување. \newline
\textbf{Опис:} Користење на алгоритми (логистичка регресија и машини со потпорни вектори) за анализа и предвидување. \newline 
\textbf{Образложение:} Преку користење на овие алгоритми, може поефикасно да ги најдеме значајните фактори кои влијаат во вредностите. \newline 

\section*{8. Опис на експериментот}
\textbf{Метод:}  
Логистичка регресија и SVM Анализа на податоци собрани преку анкети со 5,000 записи за вработени ширум светот и база на податоци на вработени преку далечина.  

\textbf{Релевантност:}  
Ова истражување е релевантно бидејќи работата од далечина стана секојдневие во денешно време. Разбирањето на факторите што влијаат на менталното здравје на вработените може да им помогне на организациите да создадат подобри политики за поддршка и зајакнување на нивната работна сила. Дополнително, примената на напредни алгоритми за предвидување овозможува објективна анализа и појасна идентификација на ризичните фактори.

\end{document}
